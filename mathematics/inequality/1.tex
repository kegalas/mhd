\documentclass[UTF8]{ctexart}
\usepackage{amsmath}
\title{数学整理}
\author{o7k40shen}
\date{\today}

\begin{document}
	\maketitle
	\tableofcontents
	\newpage
	\section{不等式}
		\subsection{均值不等式}
			$H_n$\  为调和平均数、
			$G_n$\	为几何平均数、
			$A_n$\	为算数平均数、
			$Q_n$\	为平方平均数。
			任意$x_i> 0$都成立时,有
			\[H_n=\frac{n}{\sum\limits_{i=1}^n\frac{1}{x_i}}=\frac{n}{\frac{1}{x_1}+\frac{1}{x_2}+\dots+\frac{1}{x_n}}\] 
			\[G_n=\sqrt[n]{\prod_{i=1}^{n}x_i}=\sqrt[n]{x_1 x_2 \dots x_n}\]
			\[A_n=\frac{\sum\limits_{i=1}^{n}x_i}{n}=\frac{x_1+x_2+\dots+x_n}{n}\]
			\[Q_n=\sqrt{\frac{\sum\limits_{i=1}^{n}x_i^{2}}{n}}=\sqrt{\frac{x_1^{2}+x_2^{2}+\dots+x_n^{2}}{n}}\]
			\[H_n\leq G_n\leq A_n\leq Q_n\]
			当且仅当$x_1=x_2=\dots =x_n$时取等号
		\subsection{对数平均不等式}
			$a\neq b$时,有
			\[\sqrt{ab}<\frac{a-b}{lna-lnb}<\frac{a+b}{2}\]
		\subsection{柯西不等式}
			\[\sum\limits_{i=1}^{n}a_i^{2}\sum\limits_{i=1}^{n}b_i^{2}\geq (\sum\limits_{i=1}^{n}a_i b_i)^2\]
			当且仅当$\frac{a_1}{b_1}=\frac{a_2}{b_2}=\dots =\frac{a_n}{b_n}$时取等号\\
			其中二维形式如下
			\[(a^2+b^2)(c^2+d^2)\geq (ac+bd)^2\]
			当且仅当$ad=bc$即$\frac{a}{c}=\frac{b}{d}$时取等
		\subsection{排序不等式}
			排序不等式表示如下\\
			设有两组数$a_1,a_2,\dots,a_n$和$b_1,b_2,\dots,b_n$,满足$a_1\leq a_2\leq \dots \leq a_n$且$b_1\leq b_2\leq \dots \leq b_n$
			$c_1,c_2,\dots,c_n$是$b_1,b_2,\dots,b_n$的乱序排列,则有:
			\\
			\[a_1 b_n+a_2 b_{n-1}+\dots+a_n b_1\leq a_1 c_1+a_2 c_2+\dots+a_n c_n\leq a_1 b_1+a_2 b_2+\dots+a_n b_n\]
			\\
			当且仅当$a_1=a_2=\dots=a_n$或$b_1=b_2=\dots=b_n$时取等号。
			便于记忆,常记为:
			\begin{center}
				反序和$\leq$乱序和$\leq$顺序和
			\end{center}
		\subsection{权方和不等式}
			若$a_i>0$,$b_i>0$,$m>0$,则有
			\[\sum\limits_{i=1}^{n}\frac{a_i^{m+1}}{b_i^{m}}\geq \frac{\left (\sum\limits_{i=1}^{n}a_i\right) ^{m+1}}{\left (\sum\limits_{i=1}^{n}b_i\right) ^m}\] \\
			即为
			\[\frac{a_1^{m+1}}{b_1^{m}}+\frac{a_2^{m+1}}{b_2^{m}}+\dots+\frac{a_n^{m+1}}{b_n^{m}}\geq \frac{(a_1+a_2+\dots+a_n)^{m+1}}{(b_1+b_2+\dots+b_n)^{m}}\]
			当且仅当$a_i=\lambda b_i$时取等号 \\
			其中二维形式如下\\
			对于正数$a$,$b$,$x$,$y$,有
			\[\frac{a^2}{x}+\frac{b^2}{y}\geq \frac{(a+b)^2}{x+y}\]
			当且仅当$a:b=x:y$时取等号\\
			也有
			\[\frac{a^2}{ax}+\frac{b^2}{by}=\frac{a}{x}+\frac{b}{y}\geq \frac{(a+b)^2}{ax+by}\]
			当且仅当$x=y$时取等号
		\subsection{舒尔不等式}
			$a,b,c\geq 0\quad t\in R$时,有
			\[a^t (a-b)(a-c)+b^t (b-a)(b-c)+c^t (c-a)(c-b)\geq 0\]
			当且仅当$a=b=c$,或其中两个数相等且另一个等于零时,取等号。
			特别的,当$t$为非负偶数时,此不等式对任意实数$a,b,c$成立。
		\subsection{琴生不等式}
			设$f(x)$在区间$I$上是下凸函数,则对任意$x_i\in I$及$p_i>0\quad (i=1,2,\dots,n)$,有
			\[ \frac{\sum\limits_{i=1}^{n}p_i\cdot f(x_i)}{\sum\limits_{i=1}^{n}p_i}\geq f \left (\frac{\sum\limits_{i=1}^{n}p_i\cdot x_i}{\sum\limits_{i=1}^{n}p_i} \right ) \]
			其中等号当且仅当$x_1=x_2=\dots=x_n$时成立,若$f(x)$在区间$I$上是上凸函数,则不等号反向。
	\newpage
	\section{函数}
		\subsection{拉格朗日中值定理}
			设$y=f(x)$在$[a,b]$上连续,在$(a,b)$上可导,则存在$\xi \in (a,b)$使得
			\[f^{'}(\xi)=\frac{f(b)-f(a)}{b-a}\]
		\subsection{拉格朗日乘数法}
			【例题】若正数$a,b$满足$2a+b=1$,则$\frac{a}{2-2a}+\frac{b}{2-b}$的最小值为?\\
			解:构造拉格朗日函数
			\[L(a,b,\lambda)=\frac{a}{2-2a}+\frac{b}{2-b}-\lambda(2a+b-1)\]
			令
			\[\frac{\partial L}{\partial a}=L_a=\frac{1}{2(1-a)^2}-2\lambda=0\]
			\[\frac{\partial L}{\partial b}=L_b=\frac{2}{(2-b)^2}-\lambda=0\]
			\[\frac{\partial L}{\partial \lambda}=L_\lambda=-(2a+b-1)=0\]
			联立解得
			\[a=\frac{5-3\sqrt{2}}{2},b=3\sqrt{2}-4,\lambda=\frac{1}{27-18\sqrt{2}}\]
			从而
			\[\frac{a}{2-2a}+\frac{b}{2-b}=\frac{2\sqrt{2}}{3}-\frac{1}{2}\]
			此即为所求的最小值。
		\subsection{高次韦达定理}
			设$x_1,x_2,\dots,x_n$为如下方程的根
			\[a_n x^n+a_{n-1} x^{n-1}+\dots+a_1 x+a_0=0\]
			则有
			\[x_1+x_2+\dots+x_n=-\frac{a_{n-1}}{a_n}\]
			\[x_1 x_2+x_1 x_3+\dots+x_n x_{n-1}=\frac{a_{n-2}}{a_n}\]
			\[\dots\]
			\[x_1 x_2\dots x_n=(-1)^n \frac{a_0}{a_n}\]
			其中三次的形式如下\\
			若$ax^3+bx^2+cx+d=0\ (a\neq 0)$的3个根分别为$x_1,x_2,x_3$则有
			\[x_1+x_2+x_3=-\frac{b}{a}\]
			\[x_1 x_2+x_1 x_3+x_2 x_3=\frac{c}{a}\]
			\[x_1\cdot x_2\cdot x_3=-\frac{d}{a}\]		
		\subsection{切比雪夫最佳逼近}
		\subsection{泰勒展开}
\end{document}