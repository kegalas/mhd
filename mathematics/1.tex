\documentclass[UTF8]{ctexart}
\usepackage{amsmath}
\usepackage{siunitx}
\title{数学整理}
\author{kegalas}
\date{\today}

\begin{document}
	\maketitle
	\tableofcontents
	\newpage
	\section{不等式}
		\subsection{均值不等式}
			$H_n$\  为调和平均数、
			$G_n$\	为几何平均数、
			$A_n$\	为算数平均数、
			$Q_n$\	为平方平均数。
			任意$x_i> 0$都成立时,有
			\[H_n=\frac{n}{\sum\limits_{i=1}^n\frac{1}{x_i}}=\frac{n}{\frac{1}{x_1}+\frac{1}{x_2}+\dots+\frac{1}{x_n}}\] 
			\[G_n=\sqrt[n]{\prod_{i=1}^{n}x_i}=\sqrt[n]{x_1 x_2 \dots x_n}\]
			\[A_n=\frac{\sum\limits_{i=1}^{n}x_i}{n}=\frac{x_1+x_2+\dots+x_n}{n}\]
			\[Q_n=\sqrt{\frac{\sum\limits_{i=1}^{n}x_i^{2}}{n}}=\sqrt{\frac{x_1^{2}+x_2^{2}+\dots+x_n^{2}}{n}}\]
			\[H_n\leq G_n\leq A_n\leq Q_n\]
			当且仅当$x_1=x_2=\dots =x_n$时取等号
		\subsection{对数平均不等式}
			$a\neq b$时,有
			\[\sqrt{ab}<\frac{a-b}{lna-lnb}<\frac{a+b}{2}\]
		\subsection{柯西不等式}
			\[\sum\limits_{i=1}^{n}a_i^{2}\sum\limits_{i=1}^{n}b_i^{2}\geq (\sum\limits_{i=1}^{n}a_i b_i)^2\]
			当且仅当$\frac{a_1}{b_1}=\frac{a_2}{b_2}=\dots =\frac{a_n}{b_n}$时取等号\\
			其中二维形式如下
			\[(a^2+b^2)(c^2+d^2)\geq (ac+bd)^2\]
			当且仅当$ad=bc$即$\frac{a}{c}=\frac{b}{d}$时取等
		\subsection{排序不等式}
			排序不等式表示如下\\
			设有两组数$a_1,a_2,\dots,a_n$和$b_1,b_2,\dots,b_n$,满足$a_1\leq a_2\leq \dots \leq a_n$且$b_1\leq b_2\leq \dots \leq b_n$
			$c_1,c_2,\dots,c_n$是$b_1,b_2,\dots,b_n$的乱序排列,则有:
			\\
			\[a_1 b_n+a_2 b_{n-1}+\dots+a_n b_1\leq a_1 c_1+a_2 c_2+\dots+a_n c_n\leq a_1 b_1+a_2 b_2+\dots+a_n b_n\]
			\\
			当且仅当$a_1=a_2=\dots=a_n$或$b_1=b_2=\dots=b_n$时取等号。
			便于记忆,常记为:
			\begin{center}
				反序和$\leq$乱序和$\leq$顺序和
			\end{center}
		\subsection{权方和不等式}
			若$a_i>0$,$b_i>0$,$m>0$,则有
			\[\sum\limits_{i=1}^{n}\frac{a_i^{m+1}}{b_i^{m}}\geq \frac{\left (\sum\limits_{i=1}^{n}a_i\right) ^{m+1}}{\left (\sum\limits_{i=1}^{n}b_i\right) ^m}\] \\
			即为
			\[\frac{a_1^{m+1}}{b_1^{m}}+\frac{a_2^{m+1}}{b_2^{m}}+\dots+\frac{a_n^{m+1}}{b_n^{m}}\geq \frac{(a_1+a_2+\dots+a_n)^{m+1}}{(b_1+b_2+\dots+b_n)^{m}}\]
			当且仅当$a_i=\lambda b_i$时取等号 \\
			其中二维形式如下\\
			对于正数$a$,$b$,$x$,$y$,有
			\[\frac{a^2}{x}+\frac{b^2}{y}\geq \frac{(a+b)^2}{x+y}\]
			当且仅当$a:b=x:y$时取等号\\
			也有
			\[\frac{a^2}{ax}+\frac{b^2}{by}=\frac{a}{x}+\frac{b}{y}\geq \frac{(a+b)^2}{ax+by}\]
			当且仅当$x=y$时取等号
		\subsection{舒尔不等式}
			$a,b,c\geq 0\quad t\in R$时,有
			\[a^t (a-b)(a-c)+b^t (b-a)(b-c)+c^t (c-a)(c-b)\geq 0\]
			当且仅当$a=b=c$,或其中两个数相等且另一个等于零时,取等号。
			特别的,当$t$为非负偶数时,此不等式对任意实数$a,b,c$成立。
		\subsection{琴生不等式}
			设$f(x)$在区间$I$上是下凸函数,则对任意$x_i\in I$及$p_i>0\quad (i=1,2,\dots,n)$,有
			\[ \frac{\sum\limits_{i=1}^{n}p_i\cdot f(x_i)}{\sum\limits_{i=1}^{n}p_i}\geq f \left (\frac{\sum\limits_{i=1}^{n}p_i\cdot x_i}{\sum\limits_{i=1}^{n}p_i} \right ) \]
			其中等号当且仅当$x_1=x_2=\dots=x_n$时成立,若$f(x)$在区间$I$上是上凸函数,则不等号反向。
		\subsection{绝对值不等式}
			\[||a|-|b|| \leq |a\pm b| \leq |a|+|b|\]
		\subsection{糖水不等式}
			\[\frac{b+c}{a+c}>\frac{b}{a}(a>b>0,c>0)\]
			\[\frac{b+c}{a+c}<\frac{b}{a}(b>a>0,c>0)\]
	\section{函数}
		\subsection{拉格朗日中值定理}
			设$y=f(x)$在$[a,b]$上连续,在$(a,b)$上可导,则存在$\xi \in (a,b)$使得
			\[f^{'}(\xi)=\frac{f(b)-f(a)}{b-a}\]
		\subsection{拉格朗日乘数法}
			【例题】若正数$a,b$满足$2a+b=1$,则$\frac{a}{2-2a}+\frac{b}{2-b}$的最小值为?\\
			解:构造拉格朗日函数
			\[L(a,b,\lambda)=\frac{a}{2-2a}+\frac{b}{2-b}-\lambda(2a+b-1)\]
			令
			\[\frac{\partial L}{\partial a}=L_a=\frac{1}{2(1-a)^2}-2\lambda=0\]
			\[\frac{\partial L}{\partial b}=L_b=\frac{2}{(2-b)^2}-\lambda=0\]
			\[\frac{\partial L}{\partial \lambda}=L_\lambda=-(2a+b-1)=0\]
			联立解得
			\[a=\frac{5-3\sqrt{2}}{2},b=3\sqrt{2}-4,\lambda=\frac{1}{27-18\sqrt{2}}\]
			从而
			\[\frac{a}{2-2a}+\frac{b}{2-b}=\frac{2\sqrt{2}}{3}-\frac{1}{2}\]
			此即为所求的最小值。
		\subsection{高次韦达定理}
			设$x_1,x_2,\dots,x_n$为如下方程的根
			\[a_n x^n+a_{n-1} x^{n-1}+\dots+a_1 x+a_0=0\]
			则有
			\[x_1+x_2+\dots+x_n=-\frac{a_{n-1}}{a_n}\]
			\[x_1 x_2+x_1 x_3+\dots+x_n x_{n-1}=\frac{a_{n-2}}{a_n}\]
			\[\dots\]
			\[x_1 x_2\dots x_n=(-1)^n \frac{a_0}{a_n}\]
			其中三次的形式如下\\
			若$ax^3+bx^2+cx+d=0\ (a\neq 0)$的3个根分别为$x_1,x_2,x_3$则有
			\[x_1+x_2+x_3=-\frac{b}{a}\]
			\[x_1 x_2+x_1 x_3+x_2 x_3=\frac{c}{a}\]
			\[x_1\cdot x_2\cdot x_3=-\frac{d}{a}\]		
		\subsection{泰勒展开}
			若函数$f(x)$在$x_0$存在$n$阶导数,则有
			\[f(x)=f(x_0)+\frac{f'(x_0)}{1!}(x-x_0)+\frac{f''(x_0)}{2!}(x-x_0)^2+\dots+\frac{f^{(n)} (x_0)}{n!}(x-x_0)^n+R_{n+1}\]
			上式即为函数$f(x)$在$x_0$处的泰勒展开式,其中$R_{n+1}=\frac{f^{(n+1)} (\xi)}{(n+1)!}(x-x_0)^{n+1}$(其中$\xi$介于$x$和$x_0$间)叫做拉格朗日余项。\\
			拉格朗日余项可用于证明不等式。如:\\
			$-1<x<1$时
			\[\ln(1+x)=x-\frac{x^2}{2}+\frac{x^3}{3}-\frac{x^4}{4(1+\xi)^4} (-1<\xi<1)\]
			因为$-\frac{x^4}{4(1+\xi)^4}\leq 0$,所以$\ln(1+x)\leq x-\frac{x^2}{2}+\frac{x^3}{3}$
		\subsection{极值点偏移}
			【例题】已知函数$f(x)=e^x-ax$有两个零点$x_1$和$x_2$,证明:$x_1 +x_2 >2$\\
			\[f(x)=e^x-ax=0\Leftrightarrow \frac{e^x}{x}=a\]
			令\[\varphi(x)=\frac{e^x}{x}\]
			则\[f(x_1)=f(x_2)\Leftrightarrow\varphi(x_1)=\varphi(x_2),\quad \varphi'(x)=\frac{(x-1)e^x}{x^2}\]
			因此$\varphi(x)$在$(0,1)$单减,$(1,+\infty)$单增,不妨设$0<x_1<1<x_2$\\
			则$x_1+x_2>2\Leftrightarrow x_2>2-x_1$,注意到$2-x_1>1$\\
			$\Leftrightarrow\varphi(x_2)>\varphi(2-x_1)$,注意到$\varphi(x_1)=\varphi(x_2)$\\
			则$\varphi(x_1)>\varphi(2-x_1)$,其中$0<x_1<1$\\
			令\[g(x)=\varphi(x)-\varphi(2-x), \quad 0<x<1\]
			易知
			\[g'(x)<0\]
			所以$g(x)$在$(0,1)$上单减,$g(x)>g(1)=\varphi(1)-\varphi(1)=0$\\
			即$\varphi(x)-\varphi(2-x)>0$,令$x=x_1$,Q.E.D.
		\subsection{最值函数基本定理}
			定理一:
			\[min\left\{a,b\right\}\leq\frac{a+b}{2}\leq max\left\{a,b\right\}\]
			\[min\left\{a,b\right\}\leq\sqrt{ab}\leq max\left\{a,b\right\}.(a>0,b>0)\]
			\indent 定理二:
			\[max\left\{\left|a+b\right|,\left|a-b\right| \right\}=|a|+|b|\]
			\[min\left\{\left|a+b\right|,\left|a-b\right| \right\}=||a|-|b||\]
			\indent 定理三:
			\[max\left\{|a|,|b|\right\}=\frac{|a+b|}{2}+\frac{|a-b|}{2}\]
			\[min\left\{|a|,|b|\right\}=\left|\frac{|a+b|}{2}-\frac{|a-b|}{2}\right|\]
	\section{数列}
		\subsection{不动点原理}
			【例题】求$a_1 = 1 , a_{n+1}=2a_n +1$的通项公式\\
			其特征函数为$f(x)=2x+1$,令$f(x)=x$,解得$x=-1$\\
			带入得$a_{n+1}-(-1)=2(a_n-(-1))$,即$a_{n+1}+1=2(a_n+1)$,之后根据等比数列可得$a_n=2^n -1$
	\section{组合数学}
		\subsection{容斥原理}
			建议根据韦恩图解题
		\subsection{伯努利装错信封问题}
			n封信与n个信封全部错位的组合数为
			\[f(n)=n!\left[ \frac{1}{2!}-\frac{1}{3!}+\frac{1}{4!}-\dots +(-1)^n \frac{1}{n!} \right] \]
	\section{向量}
		\subsection{极化恒等式}
			重要恒等式:$4ab=(a+b)^2-(a-b)^2$\\
			\indent 极化恒等式:$4\boldsymbol{a}\cdot \boldsymbol{b}=(\boldsymbol{a}+\boldsymbol{b})^2-(\boldsymbol{a}-\boldsymbol{b})^2$
		\subsection{分点恒等式}
			在$\triangle ABC$中,M为BC上一等分点\\
			\indent 当$\overrightarrow{BM}=\lambda \overrightarrow{MC}时$,有
			\[\overrightarrow{AM}=\frac{1}{1+\lambda}\overrightarrow{AB}+\frac{\lambda}{1+\lambda}\overrightarrow{AC}\]
		\subsection{三点共线定理}
			在平面中A、B、P三点共线的充要条件是:对于该平面内任意一点O,存在唯一的实数$x,y$使得:
			\[\overrightarrow{OP}=x\overrightarrow{OA}+y\overrightarrow{OB} \]
			且
			\[x+y=1\]
			特别的有:当P在线段AB上时,$x>0,y>0$\\
			P在线段AB之外时,$xy<0$
		\subsection{向量中值定理}
			在$\triangle ABC$中,M为BC的中点,则
			\[AB^2+AC^2=2(AM^2+BM^2)\]
			对应的向量公式有:
			\[\boldsymbol{a}^2+\boldsymbol{b}^2=2\left[\left(\frac{\boldsymbol{a}+\boldsymbol{b}}{2}\right)^2 + \left(\frac{\boldsymbol{a}-\boldsymbol{b}}{2}\right)^2 \right] \]
		\subsection{向量数乘余弦定理}
			在$\triangle ABC$中,有
			\[\overrightarrow{AB}\cdot \overrightarrow{AC}=\frac{AB^2+AC^2-BC^2}{2}\]
	\section{三角}
		\subsection{和差化积}
			\[sin\alpha+sin\beta=2sin\frac{\alpha+\beta}{2}\cdot cos\frac{\alpha-\beta}{2}\]
			\[sin\alpha-sin\beta=2cos\frac{\alpha+\beta}{2}\cdot sin\frac{\alpha-\beta}{2}\]
			\[cos\alpha+cos\beta=2cos\frac{\alpha+\beta}{2}\cdot cos\frac{\alpha-\beta}{2}\]
			\[cos\alpha-cos\beta=-2sin\frac{\alpha+\beta}{2}\cdot sin\frac{\alpha-\beta}{2}\]
		\subsection{积化和差}
			\[sin\alpha cos\beta=\frac{1}{2}\left[sin(\alpha+\beta)+sin(\alpha-\beta)\right]\]
			\[cos\alpha sin\beta=\frac{1}{2}\left[sin(\alpha+\beta)-sin(\alpha-\beta)\right]\]
			\[cos\alpha cos\beta=\frac{1}{2}\left[cos(\alpha+\beta)+cos(\alpha-\beta)\right]\]
			\[sin\alpha sin\beta=-\frac{1}{2}\left[cos(\alpha+\beta)-cos(\alpha-\beta)\right]\]
		\subsection{半角公式}
			\[sin\frac{\theta}{2}=\pm \sqrt{\frac{1-cos\alpha}{2}}\]
			\[sin\frac{\theta}{2}=\pm \sqrt{\frac{1+cos\alpha}{2}}\]
			\[tan\frac{\theta}{2}=\pm \sqrt{\frac{1-cos\alpha}{1+cos\alpha}}=\frac{sin\alpha}{1+cos\alpha}=\frac{1-cos\alpha}{sin\alpha}\]
		\subsection{辅助角公式}
			\[asin\theta\pm bcos\theta=\sqrt{a^2+b^2}sin(\theta\pm\varphi),\quad tan\varphi=\frac{b}{a}\]
	\section{统计、概率、分布}
		\subsection{期望、方差、标准差}
			数学期望:我们称$E\xi = x_1 p_1+x_2 p_2+\dots+x_n p_n$为离散型随机变量$\xi$的数学期望\\
			\indent 方差和标准差:我们称$D\xi = \sum_{i=1}^{n}(x_i-E\xi)^2 p_i$为离散型随机变量$\xi$的方差,其算数平方根$\sqrt{D\xi}=\sigma\xi$叫做离散型随机变量$\xi$的标准差\\
			\\
			定理一:
			\[E(a\xi+b)=aE\xi+b\]
			\[D(a\xi+b)=a^2 D\xi\]
			定理二:
			\[E(a\xi _1+b\xi _2)=aE\xi _1+bE\xi _2\]
		\subsection{二项分布}
			n次试验中事件A恰好发生k次
			\[P(E) ={n \choose k}p^k (1-p)^{n-k}\quad k=1,2,3,\dots\]
			我们称$\xi$服从二项分布,记作$\xi \sim B(n,p)$\\
			定理:$E\xi = np$,	$D\xi = np(1-p)$
		\subsection{正态分布}
			\[f_\xi (x)=\frac{1}{\sigma\sqrt{2\pi}}e^{-\frac{(x-\mu)^2}{2\sigma ^2}} \quad x\in R,\sigma>0\]
			记作$\xi \sim N(\mu,\sigma ^2)$\\
			性质:1.其正态曲线关于$x=\mu$对称,最高点为$\frac{1}{\sigma\sqrt{2\pi}}$\\
			2.$E\xi=\mu,D\xi=\sigma ^2$\\
			3.$\sigma$越大,正态曲线越“矮胖”,表示分布越分散,$\sigma$越小,正态曲线越“瘦高”,表示分布越集中
		\subsection{几何分布}
			在n次伯努利试验中,试验k次才得到第一次成功的机率。\\
			记为$P(\xi=k)=g(k,p)=q^{k-1}p$,其中$q=1-p$,也记为$\xi \sim GE(p)$\\
			定理:$E\xi=\frac{1}{p},D\xi=\frac{q}{p^2}$
		\subsection{超几何分布}
			它描述了从有限N个物件(其中包含M个指定种类的物件)中抽出n个物件,成功抽出该指定种类的物件的次数(不放回)\\
			记为$\eta \sim H(n,M,N)$
			\[P\left ( \eta=m\right ) = \frac{\binom{M}{m}\cdot \binom{N-M}{n-m}}{\binom{N}{n}}\quad m=0,1,2,\dots,min \left\{ n,M \right\}\]
			其中有:
			\[E(X)=\frac{nM}{N}\]
			\[D(X)=\frac{nM}{N}\left(1-\frac{M}{N}\right)\frac{N-n}{N-1}\]


	\section{方程}
		\subsection{立方和分解}
			\[a^3+b^3=(a+b)(a^2-ab+b^2)\]
		\subsection{立方差分解}
			\[a^3-b^3=(a-b)(a^2+ab+b^2)\]
		\subsection{比例性质}
			若$\frac{a}{b}=\frac{c}{d}$则有:\\
			合比性质
			\[\frac{a+b}{b}=\frac{c+d}{d}\]
			分比性质
			\[\frac{a-b}{b}=\frac{c-d}{d}\]
			合分比性质
			\[\frac{a+b}{a-b}=\frac{c+d}{c-d}\]
			等比性质
			\[\frac{a}{b}=\frac{c}{d}=\frac{a+c}{b+d}=\frac{ma+nc}{mb+nd}\]
	\section{几何}
		\subsection{射影定理}
			射影定理,又称“欧几里德定理”:在直角三角形中,斜边上的高是两条直角边在斜边射影的比例中项,每一条直角边又是这条直角边在斜边上的射影和斜边的比例中项。\\
			\\
			在$Rt\bigtriangleup  ABC$中,$\angle ABC=\ang{90}$,$BD$为斜边$AC$上的高,则有射影定理如下:
			\[BD^2=AD\cdot CD\]
			\[AB^2=AC\cdot AD\]
			\[BC^2=CD\cdot AC\]
		\subsection{阿波罗尼斯圆}
			平面内到两个定点的距离之比为常数$k(k\neq1)$的点的轨迹是圆
		\subsection{角平分线定理}
			在$\bigtriangleup ABC$中,$AM$为$\angle BAC$的角平分线,$M$在$BC$上,则有:
			\[\frac{AB}{AC}=\frac{MB}{MC}\]
		\subsection{三角形五心}
			\subsubsection{重心}
				三角形的三条边的中线交于一点,该点叫三角形的重心\\
				重心的性质:\\
				1、重心到顶点的距离与重心到对边中点的距离之比为2:1。\\
				2、重心和三角形任意两个顶点组成的3个三角形面积相等。即重心到三条边的距离与三条边的长成反比。\\
				3、重心到三角形3个顶点距离的平方和最小。\\
				4、在平面直角坐标系中,重心的坐标是顶点坐标的算术平均数,即其重心坐标为
				\[P_1=
					\begin{bmatrix}
						x_1\\
						y_1
					\end{bmatrix}
				  P_2=
					\begin{bmatrix}
						x_2\\
						y_2
					\end{bmatrix}
				  p_3=
					\begin{bmatrix}
						x_3\\
						y_3
					\end{bmatrix}
				\]
				\[
					\begin{bmatrix}
						x\\
						y
					\end{bmatrix}
					=\frac{1}{3}(P_1+P_2+P_3)=\frac{1}{3}
					\begin{bmatrix}
						x_1+x_2+x_3\\
						y_1+y_2+y_3
					\end{bmatrix}
				\]
				5. 以重心为起点,以三角形三顶点为终点的三条向量之和等于零向量。
			\subsubsection{外心}
				三角形外接圆的圆心,叫做三角形的外心。\\
				外心的性质:\\
				1、三角形的三条边的垂直平分线交于一点,该点即为该三角形的外心。\\
				2、若O是$\bigtriangleup$ABC的外心,则$\angle$BOC=2$\angle$A($\angle$A为锐角或直角)或$\angle$BOC=360°-2$\angle$A($\angle$A为钝角)。\\
				3、当三角形为锐角三角形时,外心在三角形内部;当三角形为钝角三角形时,外心在三角形外部;当三角形为直角三角形时,外心在斜边上,与斜边的中点重合。\\
				4、外心到三顶点的距离相等
			\subsubsection{垂心}
				三角形的三条高(所在直线)交于一点,该点叫做三角形的垂心。\\
				垂心的性质:\\
				1、三角形三个顶点,三个垂足,垂心这7个点可以得到6个四点圆。\\
				2、三角形外心O、重心G和垂心H三点共线,且OG:GH=1:2。(此直线称为三角形的欧拉线(Euler line))(除正三角形)\\
				3、垂心到三角形一顶点距离为此三角形外心到此顶点对边距离的2倍。\\
				4、垂心分每条高线的两部分乘积相等
			\subsubsection{内心}
				三角形内切圆的圆心,叫做三角形的内心。\\
				内心的性质:\\
				1、三角形的三条内角平分线交于一点。该点即为三角形的内心。\\
				2、直角三角形的内心到边的距离等于两直角边的和与斜边的差的二分之一。\\
				3、P为$\bigtriangleup$ABC所在空间中任意一点,点O是ΔABC内心的充要条件是:$\overrightarrow{PO}=(a×\overrightarrow{PA} +b×\overrightarrow{PB} +c×\overrightarrow{PC} )/(a+b+c)$.\\
				4、O为三角形的内心,A、B、C分别为三角形的三个顶点,延长AO交BC边于N,则有AO:ON=AB:BN=AC:CN=(AB+AC):BC\\
				5、(欧拉定理)⊿ABC中,R和r分别为外接圆为和内切圆的半径,O和I分别为其外心和内心,则$OI^2=R^2-2Rr$.\\
				6、(内角平分线分三边长度关系)
				△ABC中,0为内心,∠A 、∠B、 ∠C的内角平分线分别交BC、AC、AB于Q、P、R, 则BQ/QC=c/b, CP/PA=a/c, BR/RA=a/b.\\
				7、内心到三角形三边距离相等。
			\subsubsection{旁心}
				三角形的旁切圆(与三角形的一边和其他两边的延长线相切的圆)的圆心,叫做三角形的旁心。\\
				旁心的性质:\\
				1、三角形一内角平分线和另外两顶点处的外角平分线交于一点,该点即为三角形的旁心。\\
				2、每个三角形都有三个旁心。\\
				3、旁心到三边的距离相等。\\
		\subsection{法向量叉乘求法}
			已知不共线的两个向量$\boldsymbol{a}=(x_1,y_1,z_1)$,$\boldsymbol{b}=(x_2,y_2,z_2)$\\
			则它们所确定的平面的法向量为:
			\[\boldsymbol{n}=(y_1 z_2-z_1 y_2,z_1 x_2-x_1 z_2,x_1 y_2-y_1 x_2)\]
	\section{方法}
		\subsection{主元法}
			【例题】对任意$m\in [-1,1]$,函数$f(x)=x^2+(m-4)x+4-2m$的值恒大于零,求$x$的取值范围。\\
			\[f(x)=x^2+(m-4)x+4-2m=(x-2)m+x^2-4x+4\]
			令
			\[g(m)=(x-2)m+x^2-4x+4\]
			所以有
			\[\begin{cases}
				g(-1)=(x-2)(-1)+x^2-4x+4>0
				\\
				g(1)=(x-2)\cdot 1+x^2-4x+4>0
			\end{cases}\]
			解得$x<1 $或$ x>3$
		\subsection{裂项}
			\[\frac{1}{n(n+1)}=\frac{1}{n}-\frac{1}{n+1}\]
			\[\frac{1}{n(n+1)(n+2)}=\frac{1}{2}\left[\frac{1}{n(n+1)}-\frac{1}{(n+1)(n+2)}\right] \]
			\[\frac{1}{\sqrt{n+1}+\sqrt{n}}=\sqrt{n+1}-\sqrt{n}\]
			\[a_n=(a_n-a_{n-1})+(a_{n-1}-a_{n-2})+\dots+(a_2-a_1)+a_1\]
			\[a_n=\frac{a_n}{a_{n-1}}\bullet\frac{a_{n-1}}{a_{n-2}}\bullet\dots\bullet\frac{a_2}{a_1}\bullet a_1\]
			\[n\cdot n!=(n+1)!-n!\]
			\[C^{m}_{n}=C^{m}_{n+1}-C^{m-1}_{n}\]
			\[n(n+1)=\frac{1}{3}\left[n(n+1)(n+2)-(n-1)n(n+1)\right]\]
			\[\frac{1}{C^{1}_{n+1}C^{2}_{n}}=\frac{2}{(n+1)n(n-1)}=\frac{1}{n(n-1)}-\frac{1}{(n+1)n}\]
			\[\frac{1}{2^n(2^n-1)}=\frac{1}{2^n-1}-\frac{1}{2^n}\]
			\[\frac{n}{(n+1)!}=\frac{1}{n!}-\frac{1}{(n+1)!}\]
		\subsection{放缩}
			\[\frac{1}{n^2}=\frac{4}{4n^2}<\frac{4}{4n^2 -1}=2\left(\frac{1}{2n-1}-\frac{1}{2n+1}\right)\]
			\[\left(1+\frac{1}{n}\right)^n<1+1+\frac{1}{2\times 1}+\frac{1}{3\times 2}+\dots+\frac{1}{n(n-1)}<\frac{5}{2}\]
			\[\frac{1}{\sqrt{n+2}}<\frac{1}{\sqrt{n+2}}-\frac{1}{\sqrt{n}}\]
			\[2(\sqrt{n+1}-\sqrt{n})<\frac{1}{\sqrt{n}}<2(\sqrt{n}-\sqrt{n-1})\]
			\[\frac{2^n}{(2^n-1)^2}<\frac{2^n}{(2^n-1)(2^n-2)}=\frac{2^{n-1}}{(2^n-1)(2^{n-1}-1)}=\frac{1}{2^{n-1}-1}-\frac{1}{2^n-1}\]
			\[\frac{1}{\sqrt{n^3}}=\frac{1}{\sqrt{n\cdot n^2}}<\frac{1}{\sqrt{n(n-1)(n+1)}}=\left(\frac{1}{\sqrt{n(n-1)}}\
				-\frac{1}{\sqrt{n(n+1)}}\right)\frac{1}{\sqrt{n+1}-\sqrt{n-1}}\]
				\[=\left(\frac{1}{\sqrt{n-1}}\-\frac{1}{\sqrt{n+1}}\
				\right)\frac{\sqrt{n+1}+\sqrt{n-1}}{2\sqrt{n}}<\frac{1}{\sqrt{n-1}}-\frac{1}{\sqrt{n+1}}\]
			\[\frac{1}{\sqrt{n(n+1)}}<\sqrt{n}-\sqrt{n-1}\]
\end{document}