\documentclass[UTF8]{ctexart}
\usepackage{amsmath}
\title{数学整理}
\author{o7k40shen}
\date{\today}

\begin{document}
	\maketitle
	\tableofcontents
	\newpage
	\section{不等式}
		\subsection{均值不等式}
			$H_n$\  为调和平均数、
			$G_n$\	为几何平均数、
			$A_n$\	为算数平均数、
			$Q_n$\	为平方平均数。
			任意$x_i> 0$都成立时,有
			\[H_n=\frac{n}{\sum\limits_{i=1}^n\frac{1}{x_i}}=\frac{n}{\frac{1}{x_1}+\frac{1}{x_2}+\dots+\frac{1}{x_n}}\] 
			\[G_n=\sqrt[n]{\prod_{i=1}^{n}x_i}=\sqrt[n]{x_1 x_2 \dots x_n}\]
			\[A_n=\frac{\sum\limits_{i=1}^{n}x_i}{n}=\frac{x_1+x_2+\dots+x_n}{n}\]
			\[Q_n=\sqrt{\frac{\sum\limits_{i=1}^{n}x_i^{2}}{n}}=\sqrt{\frac{x_1^{2}+x_2^{2}+\dots+x_n^{2}}{n}}\]
			\[H_n\leq G_n\leq A_n\leq Q_n\]
			当且仅当$x_1=x_2=\dots =x_n$时取等号
		\subsection{对数平均不等式}
			$a\neq b$时,有
			\[\sqrt{ab}<\frac{a-b}{lna-lnb}<\frac{a+b}{2}\]
		\subsection{柯西不等式}
			\[\sum\limits_{i=1}^{n}a_i^{2}\sum\limits_{i=1}^{n}b_i^{2}\geq (\sum\limits_{i=1}^{n}a_i b_i)^2\]
			当且仅当$\frac{a_1}{b_1}=\frac{a_2}{b_2}=\dots =\frac{a_n}{b_n}$时取等号\\
			其中二维形式如下
			\[(a^2+b^2)(c^2+d^2)\geq (ac+bd)^2\]
			当且仅当$ad=bc$即$\frac{a}{c}=\frac{b}{d}$时取等
		\subsection{排序不等式}
			排序不等式表示如下\\
			设有两组数$a_1,a_2,\dots,a_n$和$b_1,b_2,\dots,b_n$,满足$a_1\leq a_2\leq \dots \leq a_n$且$b_1\leq b_2\leq \dots \leq b_n$
			$c_1,c_2,\dots,c_n$是$b_1,b_2,\dots,b_n$的乱序排列,则有:
			\\
			\[a_1 b_n+a_2 b_{n-1}+\dots+a_n b_1\leq a_1 c_1+a_2 c_2+\dots+a_n c_n\leq a_1 b_1+a_2 b_2+\dots+a_n b_n\]
			\\
			当且仅当$a_1=a_2=\dots=a_n$或$b_1=b_2=\dots=b_n$时取等号。
			便于记忆,常记为:
			\begin{center}
				反序和$\leq$乱序和$\leq$顺序和
			\end{center}
		\subsection{权方和不等式}
			若$a_i>0$,$b_i>0$,$m>0$,则有
			\[\sum\limits_{i=1}^{n}\frac{a_i^{m+1}}{b_i^{m}}\geq \frac{\left (\sum\limits_{i=1}^{n}a_i\right) ^{m+1}}{\left (\sum\limits_{i=1}^{n}b_i\right) ^m}\] \\
			即为
			\[\frac{a_1^{m+1}}{b_1^{m}}+\frac{a_2^{m+1}}{b_2^{m}}+\dots+\frac{a_n^{m+1}}{b_n^{m}}\geq \frac{(a_1+a_2+\dots+a_n)^{m+1}}{(b_1+b_2+\dots+b_n)^{m}}\]
			当且仅当$a_i=\lambda b_i$时取等号 \\
			其中二维形式如下\\
			对于正数$a$,$b$,$x$,$y$,有
			\[\frac{a^2}{x}+\frac{b^2}{y}\geq \frac{(a+b)^2}{x+y}\]
			当且仅当$a:b=x:y$时取等号\\
			也有
			\[\frac{a^2}{ax}+\frac{b^2}{by}=\frac{a}{x}+\frac{b}{y}\geq \frac{(a+b)^2}{ax+by}\]
			当且仅当$x=y$时取等号
		\subsection{舒尔不等式}
			$a,b,c\geq 0\quad t\in R$时,有
			\[a^t (a-b)(a-c)+b^t (b-a)(b-c)+c^t (c-a)(c-b)\geq 0\]
			当且仅当$a=b=c$,或其中两个数相等且另一个等于零时,取等号。
			特别的,当$t$为非负偶数时,此不等式对任意实数$a,b,c$成立。
		\subsection{琴生不等式}
			设$f(x)$在区间$I$上是下凸函数,则对任意$x_i\in I$及$p_i>0\quad (i=1,2,\dots,n)$,有
			\[ \frac{\sum\limits_{i=1}^{n}p_i\cdot f(x_i)}{\sum\limits_{i=1}^{n}p_i}\geq f \left (\frac{\sum\limits_{i=1}^{n}p_i\cdot x_i}{\sum\limits_{i=1}^{n}p_i} \right ) \]
			其中等号当且仅当$x_1=x_2=\dots=x_n$时成立,若$f(x)$在区间$I$上是上凸函数,则不等号反向。
	\section{函数}
		\subsection{拉格朗日中值定理}
			设$y=f(x)$在$[a,b]$上连续,在$(a,b)$上可导,则存在$\xi \in (a,b)$使得
			\[f^{'}(\xi)=\frac{f(b)-f(a)}{b-a}\]
		\subsection{拉格朗日乘数法}
			【例题】若正数$a,b$满足$2a+b=1$,则$\frac{a}{2-2a}+\frac{b}{2-b}$的最小值为?\\
			解:构造拉格朗日函数
			\[L(a,b,\lambda)=\frac{a}{2-2a}+\frac{b}{2-b}-\lambda(2a+b-1)\]
			令
			\[\frac{\partial L}{\partial a}=L_a=\frac{1}{2(1-a)^2}-2\lambda=0\]
			\[\frac{\partial L}{\partial b}=L_b=\frac{2}{(2-b)^2}-\lambda=0\]
			\[\frac{\partial L}{\partial \lambda}=L_\lambda=-(2a+b-1)=0\]
			联立解得
			\[a=\frac{5-3\sqrt{2}}{2},b=3\sqrt{2}-4,\lambda=\frac{1}{27-18\sqrt{2}}\]
			从而
			\[\frac{a}{2-2a}+\frac{b}{2-b}=\frac{2\sqrt{2}}{3}-\frac{1}{2}\]
			此即为所求的最小值。
		\subsection{高次韦达定理}
			设$x_1,x_2,\dots,x_n$为如下方程的根
			\[a_n x^n+a_{n-1} x^{n-1}+\dots+a_1 x+a_0=0\]
			则有
			\[x_1+x_2+\dots+x_n=-\frac{a_{n-1}}{a_n}\]
			\[x_1 x_2+x_1 x_3+\dots+x_n x_{n-1}=\frac{a_{n-2}}{a_n}\]
			\[\dots\]
			\[x_1 x_2\dots x_n=(-1)^n \frac{a_0}{a_n}\]
			其中三次的形式如下\\
			若$ax^3+bx^2+cx+d=0\ (a\neq 0)$的3个根分别为$x_1,x_2,x_3$则有
			\[x_1+x_2+x_3=-\frac{b}{a}\]
			\[x_1 x_2+x_1 x_3+x_2 x_3=\frac{c}{a}\]
			\[x_1\cdot x_2\cdot x_3=-\frac{d}{a}\]		
		\subsection{泰勒展开}
			若函数$f(x)$在$x_0$存在$n$阶导数,则有
			\[f(x)=f(x_0)+\frac{f'(x_0)}{1!}(x-x_0)+\frac{f''(x_0)}{2!}(x-x_0)^2+\dots+\frac{f^{(n)} (x_0)}{n!}(x-x_0)^n+R_{n+1}\]
			上式即为函数$f(x)$在$x_0$处的泰勒展开式,其中$R_{n+1}=\frac{f^{(n+1)} (\xi)}{(n+1)!}(x-x_0)^{n+1}$(其中$\xi$介于$x$和$x_0$间)叫做拉格朗日余项。\\
			拉格朗日余项可用于证明不等式。如:\\
			$-1<x<1$时
			\[\ln(1+x)=x-\frac{x^2}{2}+\frac{x^3}{3}-\frac{x^4}{4(1+\xi)^4} (-1<\xi<1)\]
			因为$-\frac{x^4}{4(1+\xi)^4}\leq 0$,所以$\ln(1+x)\leq x-\frac{x^2}{2}+\frac{x^3}{3}$
		\subsection{极值点偏移}
			【例题】已知函数$f(x)=e^x-ax$有两个零点$x_1$和$x_2$,证明:$x_1 +x_2 >2$\\
			\[f(x)=e^x-ax=0\Leftrightarrow \frac{e^x}{x}=a\]
			令\[\varphi(x)=\frac{e^x}{x}\]
			则\[f(x_1)=f(x_2)\Leftrightarrow\varphi(x_1)=\varphi(x_2),\quad \varphi'(x)=\frac{(x-1)e^x}{x^2}\]
			因此$\varphi(x)$在$(0,1)$单减,$(1,+\infty)$单增,不妨设$0<x_1<1<x_2$\\
			则$x_1+x_2>2\Leftrightarrow x_2>2-x_1$,注意到$2-x_1>1$\\
			$\Leftrightarrow\varphi(x_2)>\varphi(2-x_1)$,注意到$\varphi(x_1)=\varphi(x_2)$\\
			则$\varphi(x_1)>\varphi(2-x_1)$,其中$0<x_1<1$\\
			令\[g(x)=\varphi(x)-\varphi(2-x), \quad 0<x<1\]
			易知
			\[g'(x)<0\]
			所以$g(x)$在$(0,1)$上单减,$g(x)>g(1)=\varphi(1)-\varphi(1)=0$\\
			即$\varphi(x)-\varphi(2-x)>0$,令$x=x_1$,Q.E.D.
		\subsection{最值函数基本定理}
			定理一:
			\[min\left\{a,b\right\}\leq\frac{a+b}{2}\leq max\left\{a,b\right\}\]
			\[min\left\{a,b\right\}\leq\sqrt{ab}\leq max\left\{a,b\right\}.(a>0,b>0)\]
			\indent 定理二:
			\[max\left\{\left|a+b\right|,\left|a-b\right| \right\}=|a|+|b|\]
			\[min\left\{\left|a+b\right|,\left|a-b\right| \right\}=||a|-|b||\]
			\indent 定理三:
			\[max\left\{|a|,|b|\right\}=\frac{|a+b|}{2}+\frac{|a-b|}{2}\]
			\[min\left\{|a|,|b|\right\}=\left|\frac{|a+b|}{2}-\frac{|a-b|}{2}\right|\]
	\section{数列}
		\subsection{不动点原理}
			【例题】求$a_1 = 1 , a_{n+1}=2a_n +1$的通项公式\\
			其特征函数为$f(x)=2x+1$,令$f(x)=x$,解得$x=-1$\\
			带入得$a_{n+1}-(-1)=2(a_n-(-1))$,即$a_{n+1}+1=2(a_n+1)$,之后根据等比数列可得$a_n=2^n -1$
	\section{组合数学}
		\subsection{容斥原理}
			建议根据韦恩图解题
		\subsection{伯努利装错信封问题}
			n封信与n个信封全部错位的组合数为
			\[f(n)=n!\left[ \frac{1}{2!}-\frac{1}{3!}+\frac{1}{4!}-\dots +(-1)^n \frac{1}{n!} \right] \]
	\section{向量}
		\subsection{极化恒等式}
			重要恒等式:$4ab=(a+b)^2-(a-b)^2$\\
			\indent 极化恒等式:$4\boldsymbol{a}\cdot \boldsymbol{b}=(\boldsymbol{a}+\boldsymbol{b})^2-(\boldsymbol{a}-\boldsymbol{b})^2$
		\subsection{分点恒等式}
			在$\triangle ABC$中,M为BC上一等分点\\
			\indent 当$\overrightarrow{BM}=\lambda \overrightarrow{MC}时$,有
			\[\overrightarrow{AM}=\frac{1}{1+\lambda}\overrightarrow{AB}+\frac{\lambda}{1+\lambda}\overrightarrow{AC}\]
		\subsection{三点共线定理}
			在平面中A、B、P三点共线的充要条件是:对于该平面内任意一点O,存在唯一的实数$x,y$使得:
			\[\overrightarrow{OP}=x\overrightarrow{OA}+y\overrightarrow{OB} \]
			且
			\[x+y=1\]
			特别的有:当P在线段AB上时,$x>0,y>0$\\
			P在线段AB之外时,$xy<0$
		\subsection{向量中值定理}
			在$\triangle ABC$中,M为BC的中点,则
			\[AB^2+AC^2=2(AM^2+BM^2)\]
			对应的向量公式有:
			\[\boldsymbol{a}^2+\boldsymbol{b}^2=2\left[\left(\frac{\boldsymbol{a}+\boldsymbol{b}}{2}\right)^2 + \left(\frac{\boldsymbol{a}-\boldsymbol{b}}{2}\right)^2 \right] \]
		\subsection{向量数乘余弦定理}
			在$\triangle ABC$中,有
			\[\overrightarrow{AB}\cdot \overrightarrow{AC}=\frac{AB^2+AC^2-BC^2}{2}\]
	\section{圆锥曲线}
		\subsection{仿射变换}
	\section{三角}
		\subsection{和差化积}
			
		\subsection{积化和差}
			
		\subsection{半角公式}
		
		\subsection{诱导公式}
	\section{方法}
		\subsection{主元法}
			【例题】已知函数$f(x)=\lg \frac{1+2^x+4^x \cdot a}{a^2-a+1} $,其中$a$为常数,若当$x \in \left(-\infty ,1 \right] $时,$f(x)$有意义,求实数$a$的取值范围\\
			由$\frac{1+2^x+4^x \cdot a}{a^2-a+1}>0$,且$a^2-a+1=(a-\frac{1}{2})^2+\frac{3}{4}>0$\\
			得$1+2^x+4^x \cdot a>0$,故$a>-(\frac{1}{4^x}+\frac{1}{2^x})$\\
			当$x \in \left(-\infty ,1 \right] $时,$y=\frac{1}{4^x}$和$y=\frac{1}{2^x}$都是减函数,所以$y=-(\frac{1}{4^x}+\frac{1}{2^x})$单增,且其最大值为$-\frac{3}{4}$,所以$a$取值为$\left( -\frac{3}{4}, +\infty \right) $
\end{document}